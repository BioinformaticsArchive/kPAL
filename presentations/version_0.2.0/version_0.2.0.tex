\documentclass[slidestop]{beamer}

\title{Changes in kMer version 0.2.0}
\providecommand{\myConference}{Work discussion}
\providecommand{\myDate}{Wednesday, May 7, 2014}
\author{Jeroen F. J. Laros}
\providecommand{\myGroup}{Leiden Genome Technology Center}
\providecommand{\myDepartment}{Department of Human Genetics}
\providecommand{\myCenter}{Center for Human and Clinical Genetics}
\providecommand{\lastCenterLogo}{
  \raisebox{-0.1cm}{
    \includegraphics[height=1cm]{lgtc_logo}
    %\includegraphics[height=0.7cm]{ngi_logo}
  }
}
\providecommand{\lastRightLogo}{
  %\includegraphics[height=0.7cm]{nbic_logo}
  %\includegraphics[height=0.8cm]{nwo_logo_en}
  %\hspace{1.5cm}\includegraphics[height=0.7cm]{gen2phen_logo}
}

\usetheme{lumc}

\begin{document}

% This disables the \pause command, handy in the editing phase.
%\renewcommand{\pause}{}

% Make the title page.
\bodytemplate

% First page of the presentation.
\section{Introduction}
\subsection{$k$-mer programming library and toolkit}
\begin{pframe}
  Command line interface for the analysis of $k$-mer profiles.
  \bigskip
  \pause

  Stored profiles can be:
  \begin{itemize}
    \item Loaded.
    \item Saved.
    \item Manipulated.
    \item Merged.
    \item Compared.
    \item \ldots
  \end{itemize}

  \vfill
  \permfoot{https://git.lumc.nl/j.f.j.laros/k-mer}
\end{pframe}

\subsection{Counting $k$-mers}
\begin{pframe}
  We count all occurrences of substrings of length $k$.
  \pause

  \begin{figure}
    \colorbox{white}{
      \includegraphics[width = \textwidth]{k-mer}
    }
    \vspace{-0.5cm}
    \caption{A profile of $3$-mer counts.} \label{fig:kmerprofile}
  \end{figure}

  In Figure~\ref{fig:kmerprofile} we see a part of $3$-mer counts; \bt{AAA}
  occurs $10$ times, \bt{AAC} occurs 20 times, etc.
\end{pframe}

\section{New in version 0.2.0}
\subsection{Overview}
\begin{pframe}
  Changes to the interface:
  \begin{itemize}
    \item Version information.
    \item Positional required parameters.
    \begin{itemize}
      \item Overwrite protection.
    \end{itemize}
    \item Keyword selection of predefined functions.
    \item Custom functions on the command line.
  \end{itemize}
  \bigskip
  \pause

  Changes to the library:
  \begin{itemize}
    \item Now fully pep8 compliant.
    \item Changed to Sphinx API documentation.

  \end{itemize}
\end{pframe}

\subsection{Changes to the interface}
\begin{pframe}
  Positional required parmeters.
  \bigskip

  \begin{lstlisting}[caption={Old style parameters.}]
    kMer index -o output.9 -i input.fa -k 9
  \end{lstlisting}

  \begin{lstlisting}[caption={New style parameters.}]
    kMer index input.fa output.9 9
  \end{lstlisting}
  \pause

  Note that a dash ``\lstinline{-}'' can be used for standard input / output.
  \medskip

  \begin{lstlisting}[caption={Piping and redirecting.}]
    cat input.fa | kMer index - - 9 > output.9
  \end{lstlisting}
\end{pframe}

\subsection{Keyword selection of predefined functions}
\begin{pframe}
  \begin{lstlisting}[caption={Old style function selection.}]
    kMer diff -i input_1.9 input_2.9 -P 1
  \end{lstlisting}
  
  \begin{lstlisting}[caption={New style function selection.}]
    kMer diff -P diff-prod input_1.9 input_2.9
  \end{lstlisting}

  The help function will show the list of options.
  \bigskip
  \pause

  Available for:
  \begin{itemize}
    \item Summary function (smoothing).
    \item Pairwise distance function (comparison).
  \end{itemize}
\end{pframe}

\subsection{Custom functions on the command line}
\begin{pframe}
  Adding new functions takes a lot of time.
  \begin{itemize}
    \item Implementation.
    \item Testing.
    \item Releasing new version.
    \item Installing new version.
  \end{itemize}
  \bigskip
  \pause

  Custom functions can be convenient.
  \medskip

  \begin{lstlisting}[caption={Custom function selection.}]
    kMer diff --pairwise-function "x, y: x + y" in_1 in_2
  \end{lstlisting}
\end{pframe}

\subsection{Custom merging}
\begin{pframe}
  Instead of really merging two profiles (the sum of the counts):
  \begin{itemize}
    \item Take the maximum if one value are non-zero.
    \item First value is the second is non-zero.
    \item First value if the second is zero.
  \end{itemize}
  \bigskip
  \pause

  There is also the possibility to define your own function here.
  \medskip

  \begin{lstlisting}[caption={Custom function selection.}]
    kMer merge --merge-function "x, y: x * y" in_1 in_2 out
  \end{lstlisting}
\end{pframe}

\section{Questions?}
\lastpagetemplate
\begin{pframe}
  \begin{center}
  \end{center}
\end{pframe}

\end{document}
